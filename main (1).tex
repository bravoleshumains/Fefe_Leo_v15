\documentclass{article}

% Language setting
\usepackage[french]{babel}

% Set page size and margins
\usepackage[letterpaper,top=2cm,bottom=2cm,left=3cm,right=3cm,marginparwidth=1.75cm]{geometry}

% Useful packages
\usepackage{tikz}
\usepackage{pgfplots}

\usepackage{amsmath}
\addto\captionsfrench{\def\proofname{Preuve}}
\usepackage{amsfonts}
\usepackage{amssymb}
\usepackage{graphicx}
\usepackage[colorlinks=true, allcolors=blue]{hyperref}
\usepackage[thmmarks]{ntheorem}
\usepackage{mdframed}
\usepackage{xcolor}
\usepackage[T1]{fontenc}

% Environnements avec compteur unique
\newmdtheoremenv[
  backgroundcolor=green!10,
  linecolor=green!70!black,
  linewidth=2pt,
  topline=false,
  bottomline=false,
  rightline=false,
  leftmargin=0pt,
  innerleftmargin=10pt,
  innerrightmargin=10pt,
  innertopmargin=6pt,
  innerbottommargin=6pt,
  skipabove=\topsep,
  skipbelow=\topsep,
]{definition}{Définition}

\newmdtheoremenv[
  backgroundcolor=red!10,
  linecolor=red!70!black,
  linewidth=2pt,
  topline=false,
  bottomline=false,
  rightline=false,
  leftmargin=0pt,
  innerleftmargin=10pt,
  innerrightmargin=10pt,
  innertopmargin=6pt,
  innerbottommargin=6pt,
  skipabove=\topsep,
  skipbelow=\topsep,
]{proposition}{Proposition}

\newmdtheoremenv[
  backgroundcolor=orange!30,
  linecolor=orange!80!black,
  linewidth=2pt,
  topline=false,
  bottomline=false,
  rightline=false,
  leftmargin=0pt,
  innerleftmargin=10pt,
  innerrightmargin=10pt,
  innertopmargin=6pt,
  innerbottommargin=6pt,
  skipabove=\topsep,
  skipbelow=\topsep,
]{theorem}{Théorème}

\newmdtheoremenv[
  backgroundcolor=orange!10,
  linecolor=orange!70!black,
  linewidth=2pt,
  topline=false,
  bottomline=false,
  rightline=false,
  leftmargin=0pt,
  innerleftmargin=10pt,
  innerrightmargin=10pt,
  innertopmargin=6pt,
  innerbottommargin=6pt,
  skipabove=\topsep,
  skipbelow=\topsep,
]{corollary}{Corollaire}

\newmdtheoremenv[
  backgroundcolor=blue!10,
  linecolor=blue!70!black,
  linewidth=2pt,
  topline=false,
  bottomline=false,
  rightline=false,
  leftmargin=0pt,
  innerleftmargin=10pt,
  innerrightmargin=10pt,
  innertopmargin=6pt,
  innerbottommargin=6pt,
  skipabove=\topsep,
  skipbelow=\topsep,
]{methode}{Méthode}

% Lemme avec compteur unique
\newmdtheoremenv[
  backgroundcolor=yellow!10,
  linecolor=yellow!70!black,
  linewidth=2pt,
  topline=false,
  bottomline=false,
  rightline=false,
  leftmargin=0pt,
  innerleftmargin=10pt,
  innerrightmargin=10pt,
  innertopmargin=6pt,
  innerbottommargin=6pt,
  skipabove=\topsep,
  skipbelow=\topsep,
]{lemma}{Lemme}

\newtheorem{Proof}{Preuve}

\usepackage{graphicx} % Required for inserting images
\usepackage{stmaryrd}
\title{Cours Raphael}
\author{Léo ferrand}
\date{February 2023}
\begin{document}

\maketitle

\section{Calcul Formel}
Cette leçon porte sur une chose essentielle pour la suite du cursus, la capacité à manipuler les expressions.

Il y a 3 formules PRIMORDIALES à connaitre par coeur ou à savoir redémontrer la première étant:
\[ (a+b)^2 = a^2 +2ab +b^2\]

Elle se trouve en développant l'expression $(a+b)(a+b)$ ou bien en faisant un dessin, mettre au carré signifie littéralement calculer l'aire du carré de coté $a + b$. Moyen mémo-technique : La surface est en mètre carré.
\begin{center}
\begin{tikzpicture}
    % Définition des dimensions
    \def\a{3}    % Longueur a
    \def\b{1.5}  % Longueur b

    % Carré principal (a + b)²
    \draw (0,0) rectangle (\a+\b,\a+\b);
    
    % Division des régions
    \draw (\a,0) -- (\a,\a+\b);    % Ligne verticale
    \draw (0,\b) -- (\a+\b,\b);    % Ligne horizontale

    % Coloration des régions
    \fill[yellow] (0,\b) rectangle (\a,\a+\b);        % Carré a² (en haut à gauche)
    \fill[blue]   (\a,\b) rectangle (\a+\b,\a+\b);    % Rectangle ab (en haut à droite)
    \fill[blue]   (0,0) rectangle (\a,\b);            % Rectangle b·ab (en bas à gauche)
    \fill[green]  (\a,0) rectangle (\a+\b,\b);        % Carré b² (en bas à droite)

    % Étiquettes des surfaces
    \node at (0.5*\a, 0.5*\b + 0.5*\a + \b) {\huge $a^2$};       % Carré a²
    \node at (\a + 0.5*\b, 0.5*\a + \b)     {\huge $ab$};        % Rectangle ab
    \node at (0.5*\a, 0.5*\b)               {\huge $ba$};% Rectangle ba
    \node at (\a + 0.5*\b, 0.5*\b)          {\huge $b^2$};       % Carré b²

    % Étiquettes des dimensions
    \node[left] at (-0.2, 0.5*\a + \b) {\huge $a$};            % Côté gauche (a)
    \node[left] at (-0.2, 0.5*\b)      {\huge $b$};            % Côté gauche (b)
    \node[above] at (0.5*\a, \a+\b+0.3)    {\huge $a$};        % Côté supérieur (a)
    \node[above] at (\a + 0.5*\b, \a+\b+0.3) {\huge $b$};      % Côté supérieur (b)

\end{tikzpicture}
\end{center}








Essaye de trouver grâce à une des précédentes méthodes de ton choix, l'expression développée de:
\begin{itemize}
    \item $(a-b)^2$
    \item  $ a^2-b^2 $
    \item $(a+b)^3$ \textit{indice: un volume se mesure en mètre cube}
\end{itemize}

\subsection{Exercices de développement}
Grâce à ces formules, développe les expressions suivantes:

\[
\textbf{a) } (1-5x)(1+5x) \qquad 
\textbf{b) } 10(5x+1)^2 \qquad 
\textbf{c) } (25x-25x)^2\qquad
\textbf{d) } (x-a)(x-b)(x-c)\dots(x-y)(x-z)
\]

\subsection{Exercices de factorisation}

Factorise les expressions suivantes:
\[
\textbf{a) } x^2 -6x+9\qquad 
\textbf{b) } 2x^2 + 2\sqrt{2}x + 1\qquad 
\textbf{c) } 9x^2-4\qquad
\textbf{d) } (3x-1)^2
 - (2x-3)^2 - (x+2)(x-5)
\]

\section{Fonctions}
\begin{definition}
   Une fonction 
 permet d'associer à un nombre $x$ 
, un nombre unique transformé que l'on note $f(x)$.
\end{definition}

Si l'on note $y= f(x)$, on dit que $x$ est l'antécédent de $y$ (anté = avant donc antécédent est le nombre qu'était $y$ avant l'intervention de $f$).\\
$y$ lui est l'image de $x$ par $f$ (un miroir nous donne notre image).\\

Il existe plusieurs méthodes pour trouver l'image ou l'antécédent d'un nombre par rapport  à un fonction donnée.

\begin{itemize}
    \item le tableau:
   \begin{center}
    \begin{tabular}{|c|c|c|c|c|c|}
        \hline
        \( x \) & -2 & -1 & 0 & 1 & 2 \\ 
        \hline
        \( f(x) = 2x \) & -4 & -2 & 0 & 2 & 4 \\ 
        \hline
    \end{tabular}\\
\end{center}
    \item le graphique: où on trace les points ($x$,$f(x)$)
\begin{center}
    \begin{tikzpicture}
        \begin{axis}[
            axis lines=middle, % Axes centrés sur l'origine
            enlargelimits=true, % Ajuste les limites pour que la courbe soit bien visible
            xlabel={$x$}, % Label de l'axe x
            ylabel={$f(x)$}, % Label de l'axe y
            xtick={-2,-1,0,1,2}, % Graduations en x
            ytick={-4,-2,0,2,4}, % Graduations en y
            samples=100, % Nombre de points pour tracer la courbe
            domain=-2:2, % Domaine de définition de la courbe
        ]
        \addplot[blue, thick] {2*x}; % Tracé de la fonction f(x) = 2x en bleu et épais
        \end{axis}
    \end{tikzpicture}
\end{center}

\end{itemize}

\section{EXERCICE FINAL}
On considère deux fonctions \( f \) et \( g \) définies par :

\[
f(x) = x^2 - x - 6 \quad \text{et} \quad g(x) = -2x.
\]

\begin{enumerate}
    \item 
    \begin{enumerate}
        \item Montrer que l’image de 5 par la fonction \( f \) est 14.
        \item Déterminer l’antécédent de 4 par la fonction \( g \).
    \end{enumerate}
    
    Pour calculer des images de nombres par les fonctions \( f \) et \( g \), on utilise un tableur et on obtient la copie d’écran suivante :

    \[
    \begin{array}{|c|c|c|c|c|c|c|c|c|}
    \hline
     & A & B & C & D & E & F & G & H \\
    \hline
    1 & x & -4 & -3 & -2 & -1 & 0 & 1 & 2 \\
    \hline
    2 & f(x) = x^2 - x - 6 & 14 & 6 & 0 & -4 & -6 & -6 & -4 \\
    \hline
    3 & g(x) = -2x & 8 & 6 & 4 & 2 & 0 & -2 & -4 \\
    \hline
    \end{array}
    \]

    \begin{enumerate}
        \setcounter{enumii}{2}
        \item À l’aide des informations précédentes, citer deux antécédents de 14 par la fonction \( f \).
        \item Existe-t-il un nombre qui a la même image par la fonction \( f \) et par la fonction \( g \) ?
    \end{enumerate}
\end{enumerate}
\begin{enumerate}
    \setcounter{enumi}{1}
    \item 
    \begin{enumerate}
        \item Montrer que, pour tout nombre \( x \), \( f(x) \) est égal à \( (x+2)(x-3) \).
        \item Résoudre l’équation \( f(x) = 0 \).
    \end{enumerate}
\end{enumerate}

\end{document}
